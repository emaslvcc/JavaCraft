\section{Functionality Exploration} \label{section: functionality exploration}

\begin{table}[ht]
    \rowcolors{2}{gray!10}{white}
    \centering
    \caption{A table that describes functions used in javacraft}
    \begin{tabular}[t]{ccccc}
    \toprule
    \textbf{No.}&\textbf{Function Name}&\textbf{Description}\\
    \midrule
    Entry 1& \texttt{void generateWorld} & assigns integer to every tile of the world\\
    Entry 2& \texttt{void initGame} & creates world with width \texttt{worldWidth} and height \texttt{worldHeight}\\
    Entry 3& \texttt{void main} & main function\\
    Entry 4& \texttt{void startGame} & starts the game\\
    Entry 5& \texttt{void movePlayer} & moves player horizontally or vertically\\
    Entry 6& \texttt{void mineBlock} & mines block player is on if block is not air\\
    Entry 7& \texttt{String getBlockSymbol} & returns symbol of \texttt{blockType}\\
    Entry 8& \texttt{void resetWorld} & clears the world and sets player position in middle\\
    Entry 9& \texttt{void generateEmptyWorld} & generates an empty world\\
    Entry 10& \texttt{void clearScreen} & clears terminal\\
    Entry 11& \texttt{void lookAround} & prints out adjacent squares to player\\
    Entry 12& \texttt{void fillInventory} & completely fills up inventory of player\\
    Entry 13& \texttt{void displayLegend} & displays a legend of what each tile represents\\
    Entry 14& \texttt{void displayWorld} & prints out all tiles of the world\\
    Entry 15& \texttt{void displayInventory} & prints out obtained items \& crafted items\\
    Entry 16& \texttt{void loadGame} & loads the game from file \texttt{fileName}\\
    Entry 17& \texttt{void saveGame} & saves the game in file \texttt{fileName}\\
    Entry 18& \texttt{void interactWithWorld} & interacts with item player is standing on\\
    Entry 19& \texttt{void addCraftedItem} & adds item \texttt{craftedItem} to array \texttt{craftedItems} \\
    Entry 20& \texttt{void removeItemsFromInventory} & removes item \texttt{item} \texttt{count} times from inventory\\
    Entry 21& \texttt{boolean inventoryContains} & returns boolean of whether \texttt{item} is in inventory\\
    Entry 22& \texttt{void placeBlock} & places block \texttt{blockType} at player position\\
    Entry 23& \texttt{void displayCraftingRecipes} & prints out available crafting recipes\\
    Entry 24& \texttt{void craftItem} & crafts an item based on argument \texttt{recipe}\\
    Entry 25& \texttt{void craftIronIngot} & crafts an iron ingot\\
    Entry 26& \texttt{void craftStick} & crafts a stick\\
    Entry 27& \texttt{void waitForEnter} & waits for operator to press Enter\\
    Entry 28& \texttt{String getBlockTypeFromCraftedItem} & returns integer of \texttt{craftedItem}\\
    Entry 29& \texttt{String getCraftedItemFromBlockType} & returns integer of \texttt{blockType}\\
    Entry 30& \texttt{String getBlockName} & returns name of \texttt{blockType}\\
    Entry 31& \texttt{String getBlockColor} & returns color of \texttt{blockType}\\
    Entry 32& \texttt{String getCraftedItemName} & returns name of \texttt{craftedItem}\\
    Entry 33& \texttt{String getCraftedItemColor} & returns color of \texttt{craftedItem}\\
    Entry 34& \texttt{char getBlockChar} & returns char of \texttt{blockType}\\
    Entry 35& \texttt{void craftWoodenPlanks} & crafts wooden planks\\
    Entry 36& \texttt{void getCountryAndQuoteFromServer} & fetches HTTP request from and writes data to server and prints country and quote\\
    \bottomrule
    \end{tabular}
    \label{table: style 2}
\end{table}