\section{Team Overview} \label{section: overview}
Below describes what each team member did on the project:

\subsection{Long Luong} 
Project leader \\
LaTeX Document owner and composer \\
Created 1 flowchart with pseudocode: \\
1. getCountryAndQuoteFromServer \\
Finalised the Function Exploration part \\
Helped others with git issues \\
Refined some pseudocode that others have made

\subsection{Élisa Donéa}

Designed FSA for secret door with Chris \\
Initiated with the Function Exploration \\
Create 5 flowcharts with pseudocode: \\
1. fillInventory \\
2. generateEmptyWorld \\
3. generateWorld \\
4. initGame \\
5. lookAround \\

\subsection{Chris Munteanu}

Designed FSA for secret door with Élisa and created description of FSA \\
Create 5 flowcharts with pseudocode: \\
1. displayInventory \\
2. getBlockName \\
3. loadGame \\
4. removeItemsFromInventory \\
5. saveGame \\

\subsection{Alexia Raportaru}

Create 5 flowcharts with pseudocode: \\
1. placeBlock \\
2. craftItem \\
3. interactWithWorld \\
4. mineBlock \\
5. movePlayer \\
Created an impressive flowchart of the entire game along with its pseudocode explanation

\subsection{Explanation}

When we first met up with each other we were with the three of us. During the first week we got to know each other. Long proposed to be the document manager because he knows how to use LaTeX. He also created a (now deprecated) GitLab repository found at \url{https://gitlab.maastrichtuniversity.nl/I6359380/test-project/}. \\
During the second week Alexia joined the group. We met up with each other and we divided the roles. We had to create at least 16 flowcharts alongside with pseudocode. Long mentioned that most of the functions are vey easy to understand, and since he wanted a difficult one he proposed to do the function getCountryAndQuoteFromServer and he proposed the other three group members to do 5 flowcharts with their respective pseudocode. Everybody agreed and thinks it is a good idea. \\
In week 3 and later Élisa and Chris decided to work on the FSA together. Everybody in the group was new to Git except Long, so he helped out everyone set-up the Git environment and he explained to everyone how it works.  In the meantime, Alexia worked on the flowchart and the pseudocode of the entire game JavaCraft. Later, Long found out he made a mistake and that we had to create a branch on the repository \url{https://gitlab.maastrichtuniversity.nl/bcs1110/javacraft/} so he informed everybody about the mistake and that we all had to migrate to the correct branch.